\documentclass[12pt]{article}

\usepackage{sbc-template}
\usepackage{graphicx}
\usepackage{url}
\usepackage[utf8]{inputenc}
\usepackage[T1]{fontenc}
\usepackage[brazil]{babel}
\usepackage{tikz}
\usetikzlibrary{arrows.meta, positioning, shapes.geometric, shapes.misc, fit, backgrounds, calc}

\sloppy

% ── Global TikZ styles ──────────────────────────────────────────────────────
\tikzset{
  agent/.style={
    draw, rounded rectangle, minimum height=0.7cm, minimum width=1.4cm,
    fill=blue!8, font=\sffamily\scriptsize, align=center,
    inner sep=3pt
  },
  orchestrator/.style={
    agent, fill=orange!15, font=\sffamily\scriptsize\bfseries
  },
  memory/.style={
    draw, dashed, rounded corners=3pt, fill=gray!8,
    font=\sffamily\scriptsize, align=center, inner sep=4pt
  },
  external/.style={
    draw, dotted, rounded corners=3pt, fill=green!5,
    font=\sffamily\scriptsize, align=center, inner sep=3pt
  },
  memoryitem/.style={
    draw, rounded corners=2pt, fill=yellow!10,
    font=\sffamily\tiny, align=center, inner sep=2pt,
    minimum height=0.45cm
  },
  module/.style={
    draw, rounded corners=2pt, fill=blue!6,
    font=\sffamily\tiny, align=center, inner sep=2pt,
    minimum height=0.5cm, minimum width=1.3cm
  },
  memmodule/.style={
    draw, rounded corners=2pt, fill=yellow!8,
    font=\sffamily\tiny, align=center, inner sep=2pt,
    minimum height=0.45cm, minimum width=1.1cm
  },
  dataflow/.style={
    -{Stealth[length=2.5pt]}, thick, draw=blue!60!black
  },
  secondary/.style={
    -{Stealth[length=2pt]}, densely dashed, draw=gray!60
  },
  reportarrow/.style={
    -{Stealth[length=2.5pt]}, thick, draw=orange!70!black
  },
  storearrow/.style={
    -{Stealth[length=2pt]}, densely dashed, draw=violet!70
  },
}

\title{Sistemas Multi-Agentes Cognitivos Baseados em LLMs\\para Resolu\c{c}\~ao Colaborativa de Tarefas}

\author{Jean Reinhold\inst{1}}

\address{Centro de Desenvolvimento Tecnol\'ogico (CDTec)\\
  Universidade Federal de Pelotas (UFPel)\\
  Pelotas -- RS -- Brasil
  \email{jreinhold@inf.ufpel.edu.br}
}

\begin{document}

\maketitle

% ===========================================================================
% RESUMO
% ===========================================================================
\begin{resumo}
Sistemas multi-agentes baseados em Modelos de Linguagem de Grande Escala (LLMs) t\^em se destacado como abordagem promissora para a resolu\c{c}\~ao colaborativa de problemas complexos. Este trabalho analisa a arquitetura proposta por Silva et al. (2025), que introduz um sistema multi-agente conversacional cognitivo onde um agente orquestrador distribui dinamicamente tarefas entre agentes especializados equipados com m\'odulos cognitivos. Realizou-se uma an\'alise cr\'itica da arquitetura, identificando seus pontos fortes --- como a modularidade e o uso de mem\'oria compartilhada --- e suas limita\c{c}\~oes, incluindo desafios de escalabilidade e avalia\c{c}\~ao. Prop\~oem-se melhorias relacionadas \`a gest\~ao de mem\'oria e aos mecanismos de coordena\c{c}\~ao entre agentes.
\end{resumo}

% ===========================================================================
% ABSTRACT
% ===========================================================================
\begin{abstract}
Multi-agent systems powered by Large Language Models (LLMs) have emerged as a promising approach for collaborative complex problem-solving. This work analyzes the architecture proposed by Silva et al. (2025), which introduces a conversational cognitive multi-agent system where an orchestrator agent dynamically distributes tasks among specialized agents equipped with cognitive modules. A critical analysis of the architecture was conducted, identifying its strengths --- such as modularity and shared memory usage --- and its limitations, including scalability and evaluation challenges. Improvements related to memory management and inter-agent coordination mechanisms are proposed.
\end{abstract}

% ===========================================================================
% 1. INTRODU\c{C}\~AO
% ===========================================================================
\section{Introdu\c{c}\~ao}

Arquiteturas de sistemas multi-agentes baseadas em LLMs t\^em emergido como abordagem poderosa para a resolu\c{c}\~ao de problemas complexos, permitindo que m\'ultiplos agentes aut\^onomos colaborem em um ambiente compartilhado \cite{wang2024survey}. Apesar dos avan\c{c}os, desafios significativos persistem: a garantia de comunica\c{c}\~ao e coordena\c{c}\~ao fluidas, a manuten\c{c}\~ao de consist\^encia comportamental e a gest\~ao eficaz de mem\'oria em ambientes din\^amicos permanecem como quest\~oes em aberto \cite{sumers2024cognitive}.

Silva et al. \cite{silva2025llm} prop\~oem uma arquitetura multi-agente na qual um agente orquestrador distribui dinamicamente tarefas entre agentes cognitivos especializados, denominados \textit{Task-driven Agents}. O sistema utiliza mem\'oria compartilhada como mecanismo de coordena\c{c}\~ao e um paradigma conversacional flex\'ivel para comunica\c{c}\~ao entre agentes.

Este trabalho tem como objetivo analisar criticamente essa arquitetura, identificando seus pontos fortes e limita\c{c}\~oes, e propor melhorias. A Se\c{c}\~ao~2 apresenta o referencial te\'orico; a Se\c{c}\~ao~3 descreve a arquitetura analisada; a Se\c{c}\~ao~4 discute resultados e melhorias; e a Se\c{c}\~ao~5 apresenta as conclus\~oes.

% ===========================================================================
% 2. REFERENCIAL TE\'ORICO
% ===========================================================================
\section{Referencial Te\'orico}

Um agente inteligente \'e uma entidade computacional aut\^onoma capaz de perceber seu ambiente e agir sobre ele \cite{wooldridge1995intelligent}. Sistemas Multi-Agentes (SMA) estendem essa defini\c{c}\~ao ao considerar m\'ultiplos agentes interagindo em um ambiente compartilhado, onde coordena\c{c}\~ao e coopera\c{c}\~ao s\~ao essenciais \cite{weiss1999multiagent, russell2021artificial}. Sumers et al. \cite{sumers2024cognitive} prop\~oem o framework CoALA, que estrutura agentes de linguagem ao longo de tr\^es dimens\~oes --- mem\'oria, espa\c{c}o de a\c{c}\~ao e tomada de decis\~ao --- enquanto Wang et al. \cite{wang2024survey} introduzem um framework unificado com m\'odulos de perfil, mem\'oria, planejamento e a\c{c}\~ao.

No \^ambito de frameworks multi-agentes colaborativos, Qian et al. \cite{qian2024chatdev} prop\~oem o ChatDev, onde agentes LLM com pap\'eis predefinidos colaboram iterativamente no desenvolvimento de software. Zhang et al. \cite{zhang2023building} introduzem o CoELA, projetado para agentes corporificados com integra\c{c}\~ao de LLMs. Tais exemplos demonstram o potencial da integra\c{c}\~ao de LLMs com componentes cognitivos \cite{becker2024multi}, mas evidenciam limita\c{c}\~oes em estabilidade de racioc\'inio e coordena\c{c}\~ao, motivando a abordagem orientada por tarefas de Silva et al.

% ===========================================================================
% 3. ARQUITETURA DO SISTEMA ANALISADO
% ===========================================================================
\section{Arquitetura do Sistema Analisado}

\subsection{Vis\~ao Geral}

A arquitetura proposta por Silva et al. \cite{silva2025llm} organiza-se em tr\^es camadas (Figura~\ref{fig:overview}): (i)~uma camada de \textbf{intera\c{c}\~ao com o usu\'ario}, atrav\'es de uma interface conversacional; (ii)~um \textbf{Agente Orquestrador}, respons\'avel por distribuir tarefas e mediar a comunica\c{c}\~ao; e (iii)~um conjunto de \textbf{\textit{Task-driven Agents}} especializados que acessam ferramentas externas e compartilham uma \textbf{Mem\'oria Compartilhada}. Esta \'ultima funciona como um \textit{blackboard}, persistindo dados como DataFrames, fun\c{c}\~oes e objetos gerados pelos agentes.

\begin{figure}[ht]
\centering
\begin{tikzpicture}[node distance=0.4cm and 0.5cm]
  % ── User ──
  \node[agent, minimum width=1.2cm] (user) {Usu\'ario};

  % ── Orchestrator ──
  \node[orchestrator, right=0.8cm of user, minimum width=1.8cm] (orch) {Agente\\Orquestrador};

  % ── Task-driven agents ──
  \node[agent, right=1.0cm of orch, yshift=0.6cm] (ag1) {Agente 1};
  \node[agent, right=1.0cm of orch] (ag2) {Agente 2};
  \node[agent, right=1.0cm of orch, yshift=-0.6cm] (agn) {Agente $N$};

  % ── Agents group fit ──
  \begin{scope}[on background layer]
    \node[draw, dashed, rounded corners=4pt, fill=blue!3,
          fit=(ag1)(agn), inner xsep=8pt, inner ysep=8pt,
          label={[font=\sffamily\tiny\bfseries]above:Task-driven Agents}] (aggroup) {};
  \end{scope}

  % ── External tools ──
  \node[external, right=0.7cm of aggroup] (tools) {Ferramentas\\Externas};

  % ── Shared Memory (positioned below the dashed group box) ──
  \node[memory, below=0.35cm of aggroup, minimum width=3.8cm, minimum height=0.55cm] (shmem) {Mem\'oria Compartilhada};

  % ── Arrows: main flow ──
  \draw[dataflow] (user) -- (orch);
  \draw[dataflow] (orch) -- (ag1.west);
  \draw[dataflow] (orch) -- (ag2.west);
  \draw[dataflow] (orch) -- (agn.west);

  % Arrows: agents → tools
  \draw[secondary] (ag1.east) -- (tools.west |- ag1.east);
  \draw[secondary] (ag2.east) -- (tools.west |- ag2.east);
  \draw[secondary] (agn.east) -- (tools.west |- agn.east);

  % Arrows: agents ↔ memory (from group bottom to memory top)
  \draw[secondary] ($(aggroup.south)+(-0.5,0)$) -- ($(aggroup.south)+(-0.5,0) |- (shmem.north)$);
  \draw[secondary] (aggroup.south) -- (shmem.north);
  \draw[secondary] ($(aggroup.south)+(0.5,0)$) -- ($(aggroup.south)+(0.5,0) |- (shmem.north)$);
\end{tikzpicture}
\caption{Vis\~ao geral da arquitetura multi-agente proposta por Silva et al.}
\label{fig:overview}
\end{figure}

\subsection{Estrutura Cognitiva dos Agentes}

Diferentemente de abordagens que agrupam agentes por comportamento comum, Silva et al. adotam agentes orientados por tarefa (\textit{task-driven}), cada um composto por cinco m\'odulos cognitivos (Figura~\ref{fig:agent}). A \textbf{Mem\'oria H\'ibrida} \'e o hub central, integrando tr\^es tipos: \textit{Working Memory} (curto prazo), \textit{Mem\'oria Sem\^antica} (longo prazo, baseada em RAG \cite{lewis2021retrieval}) e \textit{Mem\'oria Procedural} (hist\'orico de intera\c{c}\~oes com o LLM). Os m\'odulos de \textbf{Planejamento} e \textbf{Execu\c{c}\~ao} transformam decis\~oes em a\c{c}\~oes concretas via ferramentas externas; a \textbf{Percep\c{c}\~ao} enriquece o conhecimento do agente a partir do ambiente; e o \textbf{Racioc\'inio} gerencia as intera\c{c}\~oes com o LLM e o c\'odigo determin\'istico.

\begin{figure}[ht]
\centering
\begin{tikzpicture}[node distance=0.4cm and 0.5cm]
  % ── Central: Hybrid Memory (generous size so contents don't overlap) ──
  \node[memory, minimum width=4.6cm, minimum height=2.0cm] (hybrid) {};
  \node[font=\sffamily\scriptsize\bfseries, anchor=north, yshift=-2pt] at (hybrid.north) {Mem\'oria H\'ibrida};

  % Memory sub-components (placed well below the title)
  \node[memmodule] at ($(hybrid.center)+(-1.5, -0.15)$) (wm) {Trabalho};
  \node[memmodule] at ($(hybrid.center)+(0, -0.15)$) (sem) {Sem\^antica\\(RAG)};
  \node[memmodule] at ($(hybrid.center)+(1.5, -0.15)$) (proc) {Procedural};

  % ── Surrounding modules ──
  \node[module, left=0.6cm of hybrid] (percep) {Percep\c{c}\~ao};
  \node[module, above=0.4cm of hybrid] (plan) {Planejamento};
  \node[module, right=0.6cm of hybrid] (exec) {Execu\c{c}\~ao};
  \node[module, below=0.4cm of hybrid, minimum width=1.6cm] (reason) {Racioc\'inio};

  % ── LLM and Agent Code ──
  \node[external, below left=0.3cm and 0.2cm of reason, minimum width=0.9cm] (llm) {LLM};
  \node[external, below right=0.3cm and 0.2cm of reason, minimum width=0.9cm] (code) {C\'odigo\\do Agente};

  % ── External tools box ──
  \node[external, right=0.5cm of exec] (exttools) {Ferramentas\\Externas};

  % ── Arrows ──
  \draw[dataflow] (percep) -- (hybrid);
  \draw[dataflow] (hybrid) -- (plan);
  \draw[dataflow] (plan.east) -- ++(0.3,0) |- (exec.north);
  \draw[dataflow] (hybrid) -- (exec);
  \draw[dataflow] (hybrid) -- (reason);
  \draw[secondary] (reason) -- (llm);
  \draw[secondary] (reason) -- (code);
  \draw[secondary] (exec) -- (exttools);

  % ── Agent boundary ──
  \begin{scope}[on background layer]
    \node[draw, densely dashed, rounded corners=5pt, fill=blue!2,
          fit=(percep)(plan)(exec)(reason)(hybrid)(llm)(code)(exttools),
          inner sep=8pt,
          label={[font=\sffamily\tiny\bfseries, anchor=south east]north east:Task-driven Agent}] {};
  \end{scope}
\end{tikzpicture}
\caption{Estrutura interna de um agente cognitivo (\textit{Task-driven Agent}).}
\label{fig:agent}
\end{figure}

\subsection{Aplica\c{c}\~ao Financeira}

A viabilidade da arquitetura foi demonstrada em um cen\'ario financeiro implementado em Python com LangGraph e GPT-4o, envolvendo tr\^es agentes (Figura~\ref{fig:report}): o \textbf{Agente Moderador} (MA), que atua como orquestrador; o \textbf{Agente Financeiro} (AF), que coleta dados via API \textit{yfinance}; e o \textbf{Agente de C\'odigo} (AC), que gera visualiza\c{c}\~oes com \textit{Plotly}. A coordena\c{c}\~ao segue o paradigma \textit{Report} \cite{becker2024multi}: cada agente reporta ao Moderador, que decide o pr\'oximo passo. A Mem\'oria Compartilhada armazena progressivamente os artefatos gerados, permitindo que agentes posteriores reutilizem resultados anteriores.

\begin{figure}[ht]
\centering
\begin{tikzpicture}[node distance=0.3cm and 0.3cm]
  % ── Top row: Report flow ──
  \node[agent, minimum width=1.0cm] (usr1) {Usu\'ario};
  \node[orchestrator, right=0.3cm of usr1, minimum width=0.9cm] (ma1) {MA};
  \node[agent, right=0.3cm of ma1, fill=green!8, minimum width=0.9cm] (af) {AF};
  \node[orchestrator, right=0.3cm of af, minimum width=0.9cm] (ma2) {MA};
  \node[agent, right=0.3cm of ma2, fill=violet!8, minimum width=0.9cm] (ac) {AC};
  \node[orchestrator, right=0.3cm of ac, minimum width=0.9cm] (ma3) {MA};
  \node[agent, right=0.3cm of ma3, minimum width=1.0cm] (usr2) {Usu\'ario};

  % ── Report arrows ──
  \draw[reportarrow] (usr1) -- (ma1);
  \draw[reportarrow] (ma1) -- (af);
  \draw[reportarrow] (af) -- (ma2);
  \draw[reportarrow] (ma2) -- (ac);
  \draw[reportarrow] (ac) -- (ma3);
  \draw[reportarrow] (ma3) -- (usr2);

  % ── External tool ──
  \node[external, above=0.35cm of af] (yf) {\textit{yfinance}};
  \draw[secondary] (af) -- (yf);

  % ── Shared Memory (title on top, items inside) ──
  \node[memory, below=0.8cm of ma2, minimum width=6.2cm, minimum height=1.1cm] (shmem) {};
  \node[font=\sffamily\scriptsize\bfseries, anchor=north, yshift=-2pt] at (shmem.north) {Mem\'oria Compartilhada};

  % Memory items (placed below the title)
  \node[memoryitem] at ($(shmem.center)+(-2.0,-0.1)$) (df) {DataFrames};
  \node[memoryitem] at ($(shmem.center)+(0,-0.1)$) (fn) {Fun\c{c}\~ao Python};
  \node[memoryitem] at ($(shmem.center)+(2.0,-0.1)$) (fig) {Gr\'afico Plotly};

  % Store arrows (separate y-offsets so they don't collide)
  \draw[storearrow] (af.south) -- ++(0,-0.3) -| (df.north);
  \draw[storearrow] (ac.south) -- ++(0,-0.15) -| (fn.north);
  \draw[storearrow] ($(ac.south)+(0.05,0)$) -- ++(0,-0.3) -| (fig.north);
\end{tikzpicture}
\caption{Fluxo de comunica\c{c}\~ao no paradigma \textit{Report} --- aplica\c{c}\~ao financeira.}
\label{fig:report}
\end{figure}

% ===========================================================================
% 4. RESULTADOS E DISCUSS\~OES
% ===========================================================================
\section{Resultados e Discuss\~oes}

A arquitetura apresenta contribui\c{c}\~oes relevantes: a abordagem \textit{task-driven} promove modularidade e escalabilidade; a Mem\'oria Compartilhada funciona como \textit{blackboard}, facilitando a troca de informa\c{c}\~oes; a integra\c{c}\~ao de m\'ultiplos tipos de mem\'oria alinha-se com o framework CoALA \cite{sumers2024cognitive}; e o paradigma conversacional flex\'ivel permite diferentes configura\c{c}\~oes de coordena\c{c}\~ao.

Contudo, identificam-se limita\c{c}\~oes: avalia\c{c}\~ao restrita a um \'unico cen\'ario, aus\^encia de resolu\c{c}\~ao de conflitos na Mem\'oria Compartilhada, comunica\c{c}\~ao exclusivamente mediada pelo Orquestrador e depend\^encia do GPT-4o. Como melhorias, prop\~oem-se: (i)~\textit{Knowledge Graphs} para enriquecer a Mem\'oria Sem\^antica; (ii)~protocolos de comunica\c{c}\~ao direta entre agentes; (iii)~estudos ablativos dos m\'odulos cognitivos; e (iv)~avalia\c{c}\~ao com m\'ultiplos cen\'arios e \textit{benchmarks} padronizados.

% ===========================================================================
% 5. CONCLUS\~OES
% ===========================================================================
\section{Conclus\~oes}

Este trabalho analisou criticamente a arquitetura multi-agente cognitiva de Silva et al. \cite{silva2025llm}, identificando como contribui\c{c}\~oes-chave a modularidade proporcionada por agentes orientados por tarefa e a integra\c{c}\~ao de m\'ultiplos tipos de mem\'oria inspirada em arquiteturas cognitivas \cite{sumers2024cognitive}. As limita\c{c}\~oes apontadas --- avalia\c{c}\~ao restrita, aus\^encia de resolu\c{c}\~ao de conflitos na mem\'oria compartilhada e comunica\c{c}\~ao exclusivamente mediada --- indicam dire\c{c}\~oes de pesquisa futuras, como a integra\c{c}\~ao de \textit{Knowledge Graphs}, o desenvolvimento de \textit{benchmarks} para arquiteturas multi-agentes cognitivas e a investiga\c{c}\~ao de paradigmas alternativos de comunica\c{c}\~ao entre agentes.

% ===========================================================================
% REFER\^ENCIAS
% ===========================================================================
\bibliographystyle{sbc}
\bibliography{references}

\end{document}
